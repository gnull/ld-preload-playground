\documentclass[a4paper,titlepage]{article}
\usepackage[T1,T2A]{fontenc}
\usepackage[utf8]{inputenc}
\usepackage[russian]{babel}
\usepackage{url}
%% \usepackage{graphicx}
\usepackage{listings}

\begin{document}

\title{Лабораторная Работа
<<Знакомство с динамически подключаемыми библиотеками и загрузчиком \texttt{ld.so} в Linux>>}
\author{Олейников Иван \\ \texttt{ivan.oleynikov95@gmail.com}}
\date{\today}
\maketitle

\tableofcontents

\lstset{
  language=bash,
  frame=single,
  basicstyle=\scriptsize\ttfamily,
  breakatwhitespace=true,
  breaklines=true
}

\section{Введение}

В следующих секциях будут коротко освещены некоторые темы, понимание которых
может пригодиться при выполнении лабораторной работы. В последней секции
приведены ссылки для более основательного изучения.

\subsection{Динамические библиотеки}

Объектный файл -- файл с объектным кодом, включающим в себя скомпилированные в
машинный код функции, значения переменных. Обычно объектный файл не может быть
вызван как программа, и используется при линковке для создания программы либо
статической/динамической библиотеки.

Динамическая билиотека -- объектный файл, который может быть связан (слинкован,
загружен) с программой во время исполенения ( в отличие от статических
библиотек, которые связываются с программой во время компиляции).

При компиляции для каждой программы задается список динамических библиотек (если
такие есть), от которых эта программа зависит. Динамические библиотеки так же
могут зависеть от других библиотек.

\subsection{\texttt{ld.so}}

Перед запуском программы стартует динамический линковщик \texttt{ld.so}, его
задача -- найти и загрузить в память динамические библиотеки, от которых зависит
программа (вместе с их зависимостями). После этого линковщик стартует саму
программу, которая сразу же сможет воспользоваться функциями/переменными из
требуемых ею библиотек.

\texttt{ld.so} меняет свое поведение в зависимости от некоторых переменных
окружения, например:

\begin{itemize}
  \item \texttt{LD\_LIBRARY\_PATH} -- список путей, по которым \texttt{ld.so}
    будет искать динамические библиотеки
  \item \texttt{LD\_DEBUG}, \texttt{LD\_VERBOSE} -- вывод отладочной информации о
    работе \texttt{ld.so}
  \item \texttt{LD\_PRELOAD} -- список динамических бибилиотек, которые будут
    быть загружены до библиотек, которые требуются программой. Библиотеки,
    обозначенные в \texttt{LD\_PRELOAD} могут заменять символы
    (функции/переменные) из требуемых программой библиотек на свои, просто
    экспортируя их. Пример: Программа A требует для своей работы функцию F из
    библиотеки B (библиотека B прописана как зависимость в программе A). Так же
    есть библиотека C со своей функцией F. Если мы вызовем программу A с
    \texttt{LD\_PRELOAD=C}, загрузятся обе библиотеки B и C. Но при попытке
    вызвать из A функцию F, будет вызвана F из C, а не F из B.
\end{itemize}

\texttt{LD\_PRELOAD} можно использовать для изменения поведения программ,
заменяя функции стандартной библиотеки Си (e.g \texttt{fopen, open, printf,
  opendir, readdir} etc), которыми пользуются большинство программ. Эту
возможность применяют во многих утилитах, таких как:

\begin{itemize}
  \item \texttt{stdbuf} из \texttt{GNU coreutils}. \texttt{stdbuf} вызывает
    переданную ей аргументом программу, с измененными настройками буферизации
    стандартных потоков ввода-вывода(\texttt{stdin, stdout, stderr}).
  \item \texttt{fakeroot}. \texttt{fakeroot} вызывает программу так, чтобы ей
    казалось, что она имеет root права и может создавать файлы, устанавливая их
    владельцем root. В рельности владельцем создаваемых файлов становится тот,
    кто запустил \texttt{fakeroot}. Это может быть полезно, если пользователю, не
    имеющему прав root, требуется создать tar архив с файлами, принадлежащими
    root.
\end{itemize}

Пример написания библиотеки и загрузки ее через \texttt{LD\_PRELOAD} см. в Архиве.

\subsection{Полезные ссылки}

Книги по системному программированию в Linux, в них можно почитать об объектных
файлах, динамических библиотеках и о многом другом:

\begin{itemize}
  \item Michael Kerrisk <<The Linux Programming Interface>>
  \item W. Richard Stevens, Stephen A. Rago <<Advanced Programming in the UNIX
    Environment>>. Есть русский перевод, называется <<UNIX. Профессиональное
    программирование>>
  \item Robert Love <<Linux System Programming>>. Или по-русски
    <<Linux. Системное программирование>>
\end{itemize}

Руководства в Linux (man pages). Чтобы посмотреть руководство \texttt{xxx(y)},
нужно вызвать \texttt{man y xxx}.

\begin{itemize}
  \item \texttt{ld(1)}
  \item \texttt{ld.so(8)}
  \item \texttt{gcc(1)}
  \item \texttt{stdbuf(1)}
  \item \texttt{ltrace(1)}
  \item \texttt{fakeroot(1)}
\end{itemize}

Интересные страницы в интернете:

\begin{itemize}
\item \texttt{LD\_PRELOAD}:
  \begin{itemize}
  \item \url{http://habrahabr.ru/post/199090/}
  \item
    \url{http://rafalcieslak.wordpress.com/2013/04/02/dynamic-linker-tricks-using-ld_preload-to-cheat-inject-features-and-investigate-programs/}
  \item \url{http://xgu.ru/wiki/LD_PRELOAD}
  \end{itemize}

\item \texttt{ld.so}
  \begin{itemize}
  \item \url{https://www.ibm.com/developerworks/ru/library/l-lpic1-v3-102-3/}
  \item \url{http://www.cs.virginia.edu/~dww4s/articles/ld_linux.html}
  \item \url{http://tldp.org/HOWTO/Program-Library-HOWTO/shared-libraries.html}
  \end{itemize}
\end{itemize}

\section{Варианты Лабораторных Работ}

В следующих секциях будут описаны варианты заданий лабораторных работ. Решением
лабораторной работы должен быть исходный код (формально, допустим код на любом
языке, но проще всего будет на С/С++), который можно собрать в динамическую
библиотеку \texttt{mylib.so}. Каждое из заданий содержит подсекцию
<<Спецификация>>, которая описывает, каким должно быть поведение программы,
запущенной с \texttt{LD\_PRELOAD=/путь/к/mylib.so}.

\subsection{\texttt{ltrace}}

Написать библиотеку, которая при загрузке через \texttt{LD\_PRELOAD} будет вести
себя похожим на утилиту \texttt{ltrace} образом.

\subsubsection{Спецификафия}
Выбрать 5-7 любых функций стандартной библиотеки Си. Библиотека должна выводить
на \texttt{stderr} сообщения о вызовах программой этих функций. При этом для
программы эти функции должны работать, т.е вызовы должны доходить до стандартной
библиотеки \texttt{libc.so}.

Поведение программы не должно отличаться ничем, кроме дополнительно выводимых сообщений.

Формат выводимых сообщений -- произвольный, но сообщения должны сообщать о том,
с каким параметрами была вызвана функция и какое значение вернула.

\subsection{\texttt{fakeroot}}

Написать библиотеку, которая при загрузке через \texttt{LD\_PRELOAD} будет
предоставлять часть функционала \texttt{fakeroot}.

\subsubsection{Спецификация}

Программа должна считать, что на запущена с правами root. Т.е
\texttt{LD\_PRELOAD=/путь/к/mylib.so whoami} должно вывести \texttt{root}, в то
же время \texttt{whoami} должно вывести что-нибудь отличное от \texttt{root}.

Программа должна считать, что владелец всех файлов root. Т.е
\texttt{LD\_PRELOAD=/путь/к/mylib.so ls -l} должен показать, что все файлы
принадлежат root, а \texttt{ls -l} должен показать, что эти файлы принадлежат
другому пользователю.

\subsection{Скрытие файлов}
Написать библиотеку, которая будет скрывать от программы некоторые файлы.

\subsubsection{Спецификация}
Список файлов, которые нужно скрыть, библиотека при запуске получает из
переменной окружения или читает из файла.

Скрываемые файлы должны быть видны в выводе:
\texttt{ls /директория/со/скрываемыми/файлами}
Но их не должно быть в выводе:
\texttt{LD\_PRELOAD=/путь/к/mylib.so ls /директория/со/скрываемыми/файлами}

Так же, попытки прочесть файл (несмотря на то, что в выводе \texttt{ls} его нет)
должны возвращать ошибку. Например
\texttt{LD\_PRELOAD=/путь/к/mylib.so cat /скрываемый/файл} может вывести \texttt{cat: No such file or directory}

\end{document}
